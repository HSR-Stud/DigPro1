\section{Intensity Transformations and Spatial Filtering  \buch{p.119}}

\subsection{Basic intensity transformation functions \buch{p.122}}
General intensity transformation function:
\begin{equation}
s = T(r)
\end{equation}

\begin{itemize}
	\item Image negatives
		\begin{equation}
			s = L-1-r
		\end{equation}
	\item Log transformations
		\begin{equation}
			s = c \cdot \log{(1 + r)}
		\end{equation}
	\item Inverse log transformations
	\item Power-law (Gamma) transformations
		\begin{equation}
			s = c r^\gamma
		\end{equation}
	\item Piecewise linear transformation functions \buch{p.128}:
			\begin{itemize}
				\item contrast stretching
				\item intensity-level slicing
				\item Bit-plane slicing: \newline
					Split a image in slices for each intensity bit. All resulting planes are binary images.
					May be used for a simple compression.
			\end{itemize}
\end{itemize}


\subsection{Histogram processing \buch{p.133}}
The histogram shows the number of occurrences of a particular intensity level relative to the total number of pixels (probability of the intensity value). \\
A normalized histogram is given by $p(r_k) = n_k / MN$ for $k=0,1,2,\dots,L-1$.

\subsubsection{Histogram equalization \buch{p.134}}
The intensity levels in an image may be viewed as random variables in the interval $[0, L-1]$.

\paragraph{For continuous Values}
\begin{equation}
	s = T(r) = (L-1)\int_0^r p_r(\omega) d\omega
\end{equation}

\paragraph{For discrete Values}
\begin{equation}
	s_k = T(r_k) = (L-1) \sum_{j=0}^k p_r(r_j) = \frac{L-1}{MN} \sum_{j=0}^k n_j \qquad k = 0,1,2,\ldots, L-1
\end{equation}
where $MN$ is the total number of pixels, $n_k$ is the number of pixels that have the intensity $r_k$ and $L$ is the number of possible intensity levels in the image.
A plot of $p_r(r_k)$ versus $r_k$ is commonly referred to as a histogram.

\newpage

\subsubsection{Histogram Matching (Specification) \buch{p.140}}
If we want to highlight only some intensity levels we need other methods than histogram equalization. \\
$p_z(z)$ is the specified PDF, we wish the output image to have.

\paragraph{For continuous Values}
\begin{align}
	s	&= T(r) = (L-1) \int_0^r p_r(\omega) d\omega \label{eq:histmatch:cont1} \\
	G(z)&= (L-1) \int_0^z p_z(t) dt = s \label{eq:histmatch:cont2} \\
	z 	&= G^{-1}[T(r)] = G^{-1}(s) \label{eq:histmatch:cont3}
\end{align}

\begin{enumerate}
  \item Obtain $p_r(r)$ from the original image and use Eq.\ref{eq:histmatch:cont1} to obtain the value of $s$.
  \item Use specified PDF $p_z(z)$ in Eq.\ref{eq:histmatch:cont2} to obtain the transformation function $G(z)$
  \item Obtain the inverse transformation $z = G^{-1}(s)$
  \item Obtain the output image by first equalizing the original image using Eq.\ref{eq:histmatch:cont1}; the pixel values in this image are the $s$ values. For each pixel with value $s$ in the equalized image, perform the inverse mapping $z = G^{-1}(s)$ to obtain the corresponding pixel in the output image. $\Rightarrow$ PDF of output image is equal to specified PDF.
\end{enumerate}

\paragraph{For discrete Values}
\begin{align}
	s_k		&= T(r_k) = \frac{L-1}{MN}\sum_{j=0}^{k} n_j \qquad
      k = 0,1,2,\ldots,L-1 \label{eq:histmatch:disc1} \\
	G(z_q)	&= (L-1) \sum_{i=0}^q p_z(z_i) = s_k  \qquad
      q = 0,1,2,\ldots,L-1\label{eq:histmatch:disc2}  \\
	z_q 	&= G^{-1}(s_k)
\end{align}

\begin{enumerate}
  \item Compute the histogram $p_r(r_k)$ of the given image.
  \item Find the histogram equalization transformation (Eq.\ref{eq:histmatch:disc1}), round the resulting $s_k$
  	values to the integer range $[0, L-1]$
  \item Compute all values of the transformation function $G$ using Eq.\ref{eq:histmatch:disc2}, where $p_z(z_i)$ are the values
  	of the specified histogram. Round the values of $G$ to integers and store them in a table
  \item For every value of $s_k$ use the stored values of $G$ to find the corresponding value of $z_q$ so that $G(z_q)$ is
  	closest to $s_k$ and store these mappings from $s$ to $z$.
  \item Form the output image by first histogram-equalizing the input image and then mapping every equalized pixel value, $s_k$, of this
  	image to the corresponding value $z_q$ in the output image using the mappings found in step 4. Note: $T$ and $G^{-1}$ can also be combined $\Rightarrow r_k \rightarrow z_q$
\end{enumerate}


\subsubsection{Local Histogram Processing \buch{p.149}}
The techniques described above are easily adapted to local enhancement. The procedure is to define a neighborhood and move its center from pixel to pixel.
At each location, the histogram of the neighborhood is computed and a transformation function is obtained. This function is then used to map the intensity of the pixel centered in the neighborhood.
See Fig. 3.26(c) \buch{p.150} for local histogram equalization with a 3x3 neighborhood.

\newpage
\subsubsection{Using Histogram Statistics for Image Enhancement \buch{p.150}}
The stated formulas can be used to calculate the global mean and variance. Equations for local mean and variance in a specified neighborhood can easily be found in a similar way. \\

    mean (average intensity):
    \begin{equation}
        m = \sum\limits_{i=0}^{L-1}r_i p(r_i)
    \end{equation}

    n-th moment (intensity variance $\mu_2 = \sigma^2$):
    \begin{equation}
        \mu_n(r) = \sum\limits_{i=0}^{L-1}(r_i -m)^n p(r_i)
    \end{equation}

    sample mean:
	\begin{equation}
		m = \frac{1}{MN} \sum\limits_{x=0}^{M-1} \sum\limits_{y=0}^{N-1} f(x,y)
	\end{equation}

	sample variance:
	\begin{equation}
		\sigma^2 =  \frac{1}{MN} \sum\limits_{x=0}^{M-1} \sum\limits_{y=0}^{N-1} \left[f(x,y)-m\right]^2
	\end{equation}

This information of the picture can be used for enhancing the image. Eq. 3.3-29 \buch{p.152} is an example of image enhancement with usage of mean and variance.
\subsection{Spatial filtering \buch{p.153}}
A spatial filter consists of
\begin{enumerate}
	\item neighborhood (typically a small rectangle) of $f(x,y)$
	\item a predefined operation, filter mask $w(x,y)$
\end{enumerate}
If the operation is linear it's a \emph{linear spatial filter} otherwise it's \emph{nonlinear}.
Linear filters may be described by a correlation of a filter $w(x,y)$ and the picture $f(x,y)$. (Not a convolution!)
\begin{eqnarray}
	\text{correlation:}& w(x,y) \text{\FiveStarOpen} f(x,y) = \sum\limits_{s=-a}^{a}\sum\limits_{t=-b}^{b}w(s,t) f(x+s, y+t) \\
	\text{convolution:}& w(x,y) \bigstar f(x,y) = \sum\limits_{s=-a}^{a}\sum\limits_{t=-b}^{b}w(s,t) f(x-s, y-t)
\end{eqnarray}
Correlation and convolution can be interchanged by rotating the filter by $180^\circ$. \\

\textbf{Zero padding}\\
  Filter $w$ of size $m \times n$ $\Rightarrow$ Pad the image with a minimum of $m-1$ rows of 0's at top \& bottom and with $n-1$ columns of 0's on left \& right

\subsubsection{2D Convolution, a fast way to calculate}
Shown on an example: \\
%To calculate the 2D-Convolution, flip the Kernel in both horizontal and vertical direction. Move the center of the Kernel over the Input, multiply all overlapping Pixel and add them up.\\
	\begin{minipage}{2.1cm}
	\centering
		\begin{tabular}{|c|c|c|} \hline
			1 & 2 & 3 \\ \hline
	      	4 & 5 & 6 \\ \hline
	      	7 & 8 & 9 \\ \hline
	      \end{tabular} \\
	      Input
	\end{minipage}
	\begin{minipage}{0.3cm}
		$\bigstar$
	\end{minipage}
	\begin{minipage}{2.5cm}
	\centering
		\begin{tabular}{|c|c|c|} \hline
			\rowcolor{lightgray} -1 & -2 & -1 \\ \hline
	      	\cellcolor{lightgray} 0 & \cellcolor{gray} 0 & \cellcolor{lightgray} 0 \\ \hline
	      	\rowcolor{lightgray} 1 & 2 & 1 \\ \hline
	      \end{tabular}\\
	      Kernel
	\end{minipage}
	\begin{minipage}{0.3cm}
	\centering
			$\Rightarrow$
		\end{minipage}
	\begin{minipage}{7cm}
	\centering
		\begin{tabular}{|c|c|c|c|c|c|c|} \hline
			\cellcolor{lightgray} 1 $\cdot$ 0 & \cellcolor{lightgray} 2 $\cdot$ 0& \cellcolor{lightgray} 1$\cdot$ 0 & 0 & 0 & 0 & 0  \\ \hline
		 	\cellcolor{lightgray} 0$\cdot$ 0 & \cellcolor{gray} 0$\cdot$ 0  & \cellcolor{lightgray} 0$\cdot$ 0 & 0& 0 & 0 & 0 \\ \hline
	      	 \cellcolor{lightgray} -1$\cdot$ 0 & \cellcolor{lightgray} -2$\cdot$ 0  & \cellcolor{lightgray} -1 $\cdot$ 1 & 2 & 3 & 0 & 0\\ \hline
	      	 0 & 0 & 4 & 5 & 6 & 0 & 0 \\ \hline
	      	 0 & 0 & 7 & 8 & 9 & 0 & 0 \\ \hline
	      	 0 & 0 & 0 & 0 & 0 & 0& 0 \\ \hline
	      	 0 & 0 & 0 & 0 & 0 & 0& 0 \\ \hline
	      \end{tabular}\\
	      Step 1
	\end{minipage}
	\begin{minipage}{0.3cm}
	\centering
		=
	\end{minipage}
	\begin{minipage}{2.3cm}
	\centering
		\begin{tabular}{|c|c|c|c|c|} \hline
			-1 & -4 & -8 & -8 & -3 \\\hline
			-4 &-13 & -20 & -17 & -6 \\ \hline
	      	-6 &-18 & -24 & -18 & -6 \\ \hline
	      	4 & 13 & 20 & 17 & 6 \\ \hline
	      	7 & 22 & 32 & 26 & 9 \\ \hline
	      \end{tabular}\\
	      Output
	\end{minipage}
\\

\subsubsection{Vector Representation of Linear Filtering}
When interest lies in the characteristic response R, it the sum of products can be written as
\begin{align}
	R & = w_1z_1 + w_2z_2 + \dots + w_{mn}z_{mn} = \sum\limits_{k=1}^{mn}w_kz_k\\
	  & = \mathbf{w}^T\mathbf{z}
\end{align}
where $\mathbf{w}$ and $\mathbf{z}$ are vectors formed from the coefficients of the mask and the image.

\subsubsection{Smoothing spatial filters \buch{p.164}} \label{sec:smoothingSpatialFilters}
The output of a smoothing, linear spatial filter is the average of the pixels contained in the neighborhood of the filter mask.
Its main application is blurring and thus reducing noise in images.
\begin{equation}
g(x,y) = \frac{\sum\limits_{s=-a}^{a}\sum\limits_{t=-b}^{b}w(s,t) f(x+s, y+t)}{\sum\limits_{s=-a}^{a}\sum\limits_{t=-b}^{b}w(s,t)}
\end{equation}
\paragraph{box filter}
A filter whose coefficients are all equal:
$ \frac{1}{9} \cdot$ \begin{tabular}{|c|c|c|} \hline 1 & 1 & 1 \\ \hline 1 & 1 & 1 \\ \hline 1 & 1 & 1  \\ \hline \end{tabular}
\paragraph{gaussian (weighted average)}
Giving more importance to some pixels:
$ \frac{1}{16} \cdot$ \begin{tabular}{|c|c|c|} \hline 1 & 2 & 1 \\ \hline 2 & 4 & 2 \\ \hline 1 & 2 & 1  \\ \hline \end{tabular}
\subsubsection{Sharpening spatial filters \buch{p.175}}

This filter highlights transitions in intensity, thus sharpening the image.
It is the opposite of the smoothing filter and since they have an integrative effect it follows that sharpening filters have a derivative effect.
\begin{align}
\frac{\partial f}{\partial x} &= f(x+1) - f(x)& \text{first-order derivative} \\
\frac{\partial^2 f}{\partial x^2} &= f(x+1) + f(x-1)  -2 f(x)& \text{second-order derivative}
\end{align}

\paragraph{Second Derivative - Laplacian \buch{p.178}}
The Laplacian is a isotropic (rotation invariant) derivative operator
\begin{align}
\nabla^2f &= \frac{\partial^2 f}{\partial x^2} + \frac{\partial^2 f}{\partial y^2} \\
\nabla^2f &= f(x+1,y) + f(x-1,y) + f(x,y+1) + f(x,y-1) - 4 f(x,y)
\end{align}

\begin{tabular}{|c|c|c|} \hline
0 & 1 & 0  \\ \hline
1 &-4 & 1  \\ \hline
0 & 1 & 0  \\ \hline
\end{tabular}
Implementation with filter masks (rotation invariant for $90^\circ$ rotations) \\ \vspace{3mm}

\begin{tabular}{|c|c|c|} \hline
1 & 1 & 1 \\ \hline
1 &-8 & 1 \\ \hline
1 & 1 & 1 \\ \hline
\end{tabular}
Incorporating the diagonal directions (rotation invariant for $45^\circ$ rotations)
\\ \\
To sharpen an image we add the Laplacian to the original

\begin{equation}
g(x,y) = f(x,y) + c \left[ \nabla^2f(x,y) \right] \qquad \textrm{with } c=-1
\end{equation}


\paragraph{Unsharp Masking / Highboost Filtering \buch{p.182}}
\begin{enumerate}
\item Blur the original $f$
\item Subtract the blurred image from the original (gives the mask $g_{mask}$)
\item Add the mask to the original
\end{enumerate}
\begin{align}
g_{mask}(x,y) &= f(x,y) - \bar{f}(x,y) \\
g(x,y) &= f(x,y) + k \cdot g_{mask}(x,y)
\end{align}

with $k=1$ this filter is called unsharp masking, with $k>1$ highboost.

\paragraph{First-Order Derivative - Gradient \buch{p.184}}
The first derivative in an image is the magnitude of the gradient.
The gradient of image $f$ at $(x,y)$ is a two-dimensional \emph{vector} pointing in the direction of the greatest rate of change.
\begin{equation}
\nabla f \equiv grad(f) \equiv \begin{pmatrix}g_x\\g_y\end{pmatrix} = \begin{pmatrix} \dfrac{\partial f}{\partial x} \\ \\ \dfrac{\partial f}{\partial y} \end{pmatrix}
\end{equation}
\textbf{Gradient Image:}\\
 We are interested in the magnitude of the gradient, which is the value of the rate of change in the direction of the gradient vector.
\begin{equation}
M(x,y) = ||\nabla f|| = mag(\nabla f) = \sqrt{g_x^2 + g_y^2}
\end{equation}
\begin{itemize}
\item isotropic but not linear
\item $M(x,y)$ is called the gradient image
\item a popular (but non-isotropic) approximation is
  \[M(x,y) \approx |g_x| + |g_y|\]
\end{itemize}

The gradient can be approximated using filter masks.

\begin{itemize}
	\item Roberts cross-gradient operators
		\begin{itemize}
		\item $g_x$ = \begin{tabular}{|c|c|} \hline
		-1 & 0 \\ \hline
		 0 & 1 \\ \hline
		\end{tabular}
		\text{\FiveStarOpen} $f(x,y)$
		$\qquad g_y$ = \begin{tabular}{|c|c|} \hline
		 0 &-1 \\ \hline
		 1 & 0 \\ \hline
		\end{tabular}
		\text{\FiveStarOpen} $f(x,y)$
	\item no center of symmetry $\rightarrow$ difficult to implement
	\end{itemize}
		\item  \textbf{Sobel operators}, often used
		\begin{itemize}
		\item $g_x$ =
        \begin{tabular}{|c|c|c|} \hline
		      -1 &-2 &-1 \\ \hline
		       0 & 0 & 0 \\ \hline
		       1 & 2 & 1 \\ \hline
		    \end{tabular}
		    \text{\FiveStarOpen} $f(x,y)$
		    $\qquad g_y$ =
        \begin{tabular}{|c|c|c|} \hline
		      -1 & 0 & 1 \\ \hline
	      	-2 & 0 & 2 \\ \hline
	      	-1 & 0 & 1 \\ \hline
		    \end{tabular}
		    \text{\FiveStarOpen} $f(x,y)$
		\item 3x3 neighborhood $\rightarrow$ smoothing
		\end{itemize}
\end{itemize}

\subsection{Combining spatial enhancement methods \buch{p.191}}
