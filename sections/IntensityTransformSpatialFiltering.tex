\section{Intensity Transformations and Spatial Filtering \buch{Chapter 3}}

\subsection{Basic intensity transformation functions}
\begin{equation}
s = T(r)
\end{equation}

\begin{itemize}
	\item Image negatives
		\begin{equation}
			s = L-1-r
		\end{equation}
	\item Log transformations
		\begin{equation}
			s = c \cdot \log{(1 + r)}
		\end{equation}
	\item Inverse log transformations
	\item Power-law (Gamma) transformations
		\begin{equation}
			s = c r^\gamma
		\end{equation}
	\item Piecewise-linear transformation functions
	\item Bit-plane slicing
		Split a image in slices for each intensity bit.  All resulting plane are binary images.
		May be used for a simple compression.
	
\end{itemize}

\subsection{Histogram processing}
The histogram shows the number of occurrence of a particular intensity level relative to the total number of pixels (probability of the intensity value)
\subsubsection{Histogram equalization}

\subsection{Fundamentals of spatial filtering}
\subsubsection{Smoothing spatial filters}
\subsubsection{Sharpening spatial filters}
