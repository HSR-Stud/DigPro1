\section{Intensity Transformations and Spatial Filtering \buch{Chapter 3}}

\subsection{Basic intensity transformation functions}
\begin{equation}
s = T(r)
\end{equation}

\begin{itemize}
	\item Image negatives
		\begin{equation}
			s = L-1-r
		\end{equation}
	\item Log transformations
		\begin{equation}
			s = c \cdot \log{(1 + r)}
		\end{equation}
	\item Inverse log transformations
	\item Power-law (Gamma) transformations
		\begin{equation}
			s = c r^\gamma
		\end{equation}
	\item Piecewise-linear transformation functions
	\item Bit-plane slicing
		Split a image in slices for each intensity bit.  All resulting plane are binary images.
		May be used for a simple compression.
	
\end{itemize}


\subsection{Histogram processing}
The histogram shows the number of occurrence of a particular intensity level relative to the total number of pixels (probability of the intensity value)

\subsubsection{Histogram equalization}
The intensity level in an image may be viewed as random variables in the interval $[0, L-1]$.

\paragraph{For continuous Values}
\begin{equation}
	s = T(r) = (L-1)\int_0^r p_r(\omega) d\omega
\end{equation} 

\paragraph{For discrete Values}
\begin{equation}
	s_k = T(r_k) = (L-1) \sum_{j=0}^k p_r(r_j) = \frac{L-1}{MN} \sum_{j=0}^k n_j \qquad k = 0,1,2,\ldots, L-1
\end{equation}
where $MN$ is the total number of pixels, $n_k$ is the number of pixels that have the intensity $r_k$ and $L$ is the number of possible intensity levels in the image.
A plot of $p_r(r_k)$ versus $r_k$ is commonly referred to as a histogram.


\subsubsection{Histogram Matching (Specification)}
If we want to highlight only some intensity levels we need other methods than histogram equalization.

\paragraph{For continuous Values}
\begin{align}
	s	&= T(r) = (L-1) \int_0^r p_r(\omega) d\omega \label{eq:histmatch:cont1} \\
	G(z)&= (L-1) \int_0^z p_z(t) dt \label{eq:histmatch:cont2} \\
	z 	&= G^{-1}[T(r)] = G^{-1}(z) \label{eq:histmatch:cont3}
\end{align}

\begin{enumerate}
  \item Optain $p_r(r)$ from the original image and use Eq.\ref{eq:histmatch:cont1} to obtain the value of $s$.
  \item Use Eq.\ref{eq:histmatch:cont2} to obtain the transformation function $G(z)$
  \item Obtain the inverse transformation $z = G^{-1}(s)$
  \item Obtain the output image by first equalizing the original image using Eq.\ref{eq:histmatch:cont1}; the pixels values in this image are the $s$ values.
  	For each pixel witch value $s$ in the equalized image, perform the inverse mapping $z = G^{-1}(s)$ to obtain the corresponding pixel in the output image. 
\end{enumerate}

\paragraph{For discrete Values}
\begin{align}
	s_k		&= T(r_k) = \frac{L-1}{MN}\sum_{j=0}^{k} n_j \label{eq:histmatch:disc1} \\
	G(z_q)	&= (L-1) \sum_{i=0}^q p_z(z_i) = s_k \label{eq:histmatch:disc2}  \\
	z_q 	&= G^{-1}(s_k)	 
\end{align}

\begin{enumerate}
  \item Compute the histogram $p_r(r)$ of the given image.
  \item Find the histogram equalization transformation (Eq.\ref{eq:histmatch:disc1}), round the resulting $s_k$
  	values to the integer rage $[o, L-1]$
  \item Compute all values of the transformation function $G$ using Eq.\ref{eq:histmatch:disc2}, where $p_z(z_i)$ are the values
  	of the specified histogram. Round the values of $G$ to integers and store them in a table
  \item For every value of $s_k$ use the stored values of $G$ to find the corresponding value of $z_q$ so that $G(z_q)$ is 
  	closest to $s_k$ and store these mappings from $s$ to $z$.
  \item Form the output image by first histogram-equalizing the input image and then mapping every equalized pixel value, $s_k$, of this
  	image to the corresponding value $z_q$ in the output image using the mappings found in step 4.
\end{enumerate}


\subsubsection{Local Histogram Processing}
The techniques described above are easily adapted to local enhancement. The procedure is to define a neighborhood and move its center from pixel to pixel.
At each location, the histogram of the neighborhood is computed and a transformation function is obtained. This function is then used to map the intensity
of the pixel centered in the neighborhood.


\subsubsection{Using Histogram Statistics for Image Enhancement}



\subsection{Fundamentals of spatial filtering}
\subsubsection{Smoothing spatial filters}
\subsubsection{Sharpening spatial filters}
