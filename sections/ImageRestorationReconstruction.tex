\section{Image restoration and reconstruction \buch{p.311}}
\subsection{Noise models  \buch{p.313}}
	\subsubsection{Estimation of noise parameters}
		\begin{itemize}
			\item Select a strip of image, where the noiseless picture would be flat\\
			$\Rightarrow$ The histogram in these areas give a first hint about the noise corrupting the image 
			\item The normalized histogram is a proxy of the pdf of the noise, so the mean and variance can be estimated with (5.2-15) and (5.2-16).
			\item The shape identifies the noise model which matches the best $\Rightarrow$ $\chi^2$-Test			
		\end{itemize}
		

\subsection{Restoration in the presence of noise only - spatial filtering \buch{p.322}}
	\[
		g(x,y) = f(x,y) + \eta(x,y) \quad
		\Leftrightarrow \quad
		G(u,v) = F(u,v) + N(u,v)
	\]
	\begin{center}
		Spatial filtering (low pass) is the method of choise in situations when only addative random noise is present.
	\end{center}
	
\subsubsection{Mean Filters}
\textbf{Arithmetic mean filter}\\
This filter is a standard filter.
\begin{equation}
	\hat{f}(x,y)= \frac{1}{m  n} \sum\limits_{(s,t)\in S_{xy}}g(s,t)
\end{equation}

\textbf{Geometric mean filter}\\
Smooth similar to arithmetic mean, but less loss of detail.
\begin{equation}
	\hat{f}(x,y)= \left[\prod\limits_{(s,t)\in S_{xy}} g(s,t)\right]^\frac{1}{mn}
\end{equation}

\textbf{Harmonic mean filter}\\
Works well on slat noise, but fails for pepper noise. It does well also with other types of noise like Gaussian noise
\begin{equation}
	\hat{f}(x,y)= \frac{mn}{\sum\limits_{(s,t)\in S_{xy}} \frac{1}{g(s,t)}}
\end{equation}

\textbf{Contraharmonic mean filter}
\begin{itemize}
\item For positive Q, good for pepper noise
\item For negative Q, good for salt noise
\item For Q = -1, it is a harmonic mean filter
\item For Q = 0, it is a arithmetic mean filter\\ \\
\textbf{This filter can not simultaneously reduce salt and pepper noise}
\end{itemize}

\begin{equation}
	\hat{f}(x,y)= \frac{\sum\limits_{(s,t)\in S_{xy}} g(s,t)^{Q+1}}{\sum\limits_{(s,t)\in S_{xy}} g(s,t)^Q}
\end{equation}

\subsubsection{Order-Statistic Filters \buch{p.325}}

The output of these filters depends on the \textbf{order of the pixel} values. \\

\textbf{Median filter}\\
Quite popular, since they result in good noise suppression, without much smoothing \\
\begin{equation}
	\hat{f}(x,y)=\underset{(s,t) \in S_{xy}}{\text{median}}\{g(s,t) \}
\end{equation}

\textbf{Max filter}\\
This filter picks the maximum of the neighborhood, also it reduce the pepper noise.
\begin{equation}
	\hat{f}(x,y)=\underset{(s,t) \in S_{xy}}{\text{max}}\{g(s,t) \}
\end{equation}

\textbf{Min filter}\\
This filter picks the minimum of the neighborhood, also it reduce the salt noise.
\begin{equation}
	\hat{f}(x,y)=\underset{(s,t) \in S_{xy}}{\text{min}}\{g(s,t) \}
\end{equation}

\textbf{Midpoint filter}\\
The midpoint filter simply computes the midpoint between the maximum and the minimum values in the neighborhood. \\
It works best for randomly distributed noise, like Gaussian or uniform noise.
\begin{equation}
	\hat{f}(x,y)=\frac{1}{2} \left[ \underset{(s,t) \in S_{xy}}{\text{max}}\{g(s,t) \} + \underset{(s,t) \in S_{xy}}{\text{min}}\{g(s,t) \}\right]
\end{equation}

\textbf{Alpha-trimmed filter}\\
It is robust against outliers and it is good against "normal" noise.\\ 

Suppose that we delete the $d/2$ lowest and $d/2$ highest intensity values of $g(s,t)$ in the neighborhood $S_{xy}$. Let $g_r(s,t)$ represent the remaining $mn-d$ pixels.
\begin{itemize}
	\item $d$ can range from 0 to $mn-1$
	\item When $d=0$, then it is a arithmetic mean filter
	\item When $d=mn-1$, then the filter becomes a median filter
\end{itemize}
\begin{equation}
	\hat{f}(x,y)=\frac{1}{mn-d} \sum\limits_{(s,t)\in S_{xy}} g_r(s,t)
\end{equation}

\subsubsection{Adaptive Filter \buch{p.330}}
\textbf{Adaptive local noise reduction filter}\\
Change the filter, depending on the contents of $S_{xy}$
\begin{itemize}
	\item The local mean $m_L$ and the local variance $\sigma_{L}^2$ can be used as compact descriptors of $S_{xy}$
	\item The ratio of the variances can in theory never be lager then 1 $\Rightarrow \quad\sigma_{\eta}^2 \le \sigma_{L}^2$
	\item In practice, since the noise variance $\sigma_{\eta}^2$ needs to be estimated (schätzen), this must be enforced.\\
	$\Rightarrow$ If $\sigma_{\eta}^2 > \sigma_{L}^2$, set ratio $\frac{\sigma_{\eta}^2}{\sigma_{L}^2}$ to 1
\end{itemize}
\begin{equation}
	\hat{f}(x,y)=g(x,y) - \frac{\sigma_{\eta}^2}{\sigma_{L}^2} \left[g(x,y)-m_L \right]
\end{equation}


\textbf{Adaptive median filter \buch{p.332}}\\



\subsection{Restoration in the presence of noise only - frequency domain filtering for periodic noise \buch{p.335}}
\subsubsection{Bandreject, Bandpass Filters \buch{p.337}}
\subsubsection{Notch Filter \buch{p.337}}
\subsubsection{Optimum Notch Filtering \buch{p.338}}
	\begin{itemize}
		\item Optimally, such that the resulting variance is minimized
	\end{itemize}
	
\subsection{Linear, Position-Invariant Degradations}
	\[
		g(x,y) = H[f(x,y)] + \eta(x,y) \quad
		\Leftrightarrow \quad
		G(u,v) = H(u,v)F(u,v) + N(u,v)
	\]
	
\subsubsection{Estimating the degradation function \buch{p.346}}
\textbf{By image observation}
	\begin{itemize}
		\item Gather information from the image itself
		\item High contrast part of image is restored by hand and then degradtion function is estimated $\Rightarrow$ labor intensive
	\end{itemize}
\textbf{By experminentaion}
	\begin{itemize}
		\item Image acquisition system can be used to estimate the degradation function\\
		$\Rightarrow$ Obtain the impulse response of the degradiation by imaging an impulse (small dot of light)
	\end{itemize}

\textbf{By modeling}\\
	\begin{itemize}
		\item Preferred way, since general insights into degradation process can be found and it can be fully automated
		\item The Mathematical model comes for example from the physics of atmospheric turbulence. 
			\begin{equation}
				H(u,v) = e^{-k(u^2+v^2)^{\frac{5}{6}}}
			\end{equation}
	\end{itemize}


To create a \textbf{motion blurring}:

	\begin{equation}
		H(u,v) = \frac{T}{\pi (ua + vb)}\sin[\pi (ua + vb)] e^{-j \pi(ua + vb)}
	\end{equation}

	\begin{center}
		motion in x-dim. : $x_0(t) = \frac{a \cdot t}{T} \qquad \qquad$ 
		motion in y-dim. : $y_0(t) = \frac{b \cdot t}{T}$
	\end{center}

\subsection{Inverse filtering \buch{p.351}}

\begin{itemize}
	\item Clearly, even though we might know $H(u,v)$ perfectly, the noise makes it impossible to recover $f(x,y)$
	\item Furthermonre, if H(u,v) is close to zero, $\frac{N(u,v)}{H(u,v)}$ will dominate the result, rendering it useless.
	\item One trick is to focus on $H(u,v)$ near the origin, where it tends to be lager. $\Rightarrow$ Use of lowpass
\end{itemize}

\begin{equation}
	\hat{F}(x,y)= F(u,v) + \frac{N(u,v)}{H(u,v)}
\end{equation}


\subsection{Minimum Mean Square Error Filtering (Wiener) \buch{p.352}}
The goal is a minimum mean square error estimate. Image and noise are considered random variables. The goal is to deal with the degradation and 
\begin{equation}
	e^2 = E\{(f- \hat{f})^2 \} \qquad \hat{f} \text{ is the estimate}
\end{equation}

\begin{itemize}
	\item noise and image are uncorrelated
	\item one of both is zero mean
\end{itemize}

\begin{equation}
	\hat{F}(u,v) = \left[ \frac{1}{H(u,v)} \frac{|H(u,v)|^2}{|H(u,v)|^2 +K} \right] G(u,v) \qquad \textrm{with } K = \frac{S_{\eta}(u,v)}{S_f(u,v)} \approx const
\end{equation}\\

The parameter $K$ is evalueted iteratevely, often using humans to judge the result. The function $H(u,v)$ is known.
 Wiener filter are optimal in the mean square error in the statistic sense, which means on average over an ensemble of image.

\subsection{Constrained least squares filtering \buch{p.357}}
While the Wiener filter is powerful, the power spectral densities of the noise and the image need to be know, or a fixed ratio must be a good approximation. With the contrained least squares filter, only the noise mean and variance need to be known.

The goal is to find the smoothest image, witch satisfies the original equation, in some meaningful form.
 
\begin{equation}
	\hat{F}(u,v) = \left[ \frac{H^*(u,v)}{|H(u,v)|^2 + \gamma |P(u,v)|^2} \right] G(u,v)
	\label{eq:ConstrainedFilter}
\end{equation}\\

\begin{itemize}
	\item $P(u,v)$ is the Fourier transform of the Laplacian operator, see  Eq.~\ref{equ:Laplacian_Freq_Domain}
	\item $\gamma$ can be iteratively adjusted by a human operator or automatically.
\end{itemize}

\textbf{To automatically adjust the $\gamma$ use the Eq.~\ref{eq:error}}

\begin{equation}
	||\mathbf{r}||^{2} = \mathbf{r}^{T}\mathbf{r} \quad \textrm{with} \quad\mathbf{r} = \mathbf{g}-\mathbf{H} \mathbf{\hat{f}}	
	\label{eq:rCalculate}
\end{equation}

\begin{equation}
	||\bm\eta||^2= MN[\sigma_{\eta}^2 + m_{\eta}^2]	
\end{equation}

\begin{equation}
	||\mathbf{r}||^2 = ||\bm\eta||^2 \pm a
	\label{eq:error}\\
\end{equation}
\begin{enumerate}
	\item Specify an inital value of $\gamma$
	\item Compute $||\mathbf{r}||^2$ with Eq.~\ref{eq:rCalculate}, or  Eq.~\ref{eq:rCalculateFreq}
	\item Stop if Eq.~\ref{eq:error} is statisfied, otherwise return to step 2 after increasing $\gamma$ if $||\mathbf{r}||^2 < ||\bm\eta||^2-a$ or  decreasing $\gamma$ if  $||\mathbf{r}||^2 > ||\bm\eta||^2+a$. Use the new value of $\gamma$ in Eq.~\ref{eq:ConstrainedFilter} to recompute the optimum estimate $\hat{F}(u,v)$
\end{enumerate}

\textbf{Calculation of $||\mathbf{r}||^2$ in frequency domain}
	\begin{equation}
		R(u,v) = G(u,v) - H(u,v)\hat{F}(u,v)
	\end{equation}
	\begin{equation}
		||\mathbf{r}||^2 = \sum\limits_{x=0}^{M-1} \sum\limits_{y=0}^{N-1} r^{2}(x,y) \quad \textrm{where } r(x,y) \textrm{ is inverse of } R(u,v) 
		\label{eq:rCalculateFreq}
	\end{equation}

\subsection{Geometric mean filter \buch{p.361}}
The form of the Wiener filter can be generalized to the so called geometric mean filter.

\begin{equation}
	\hat{F}(u,v) =  \left[ \frac{H^*(u,v)}{|H(u,v)|^2} \right]^{\alpha}  \left[ \frac{H^*(u,v)}{|H(u,v)|^2 + \beta \left[ \frac{S_\eta(u,v)}{S_f(u,v)}\right]} \right]^{1-\alpha} G(u,v)
\end{equation}

\begin{itemize}
	\item $\alpha$ and $\beta$ are positive constants
	\item $\alpha = 1$, this is simply the inverse filter
	\item $\alpha=0$ and $\beta=1$, this is the Wiener filter
	\item $\alpha=0$ and $0 < \beta < 1$, this is the parametric Wiener filter
	\item $\alpha=0.5$ and $\beta=1$, this is the geometic mean of the inverse filter and the Wiener filter, also called spectrum equalization filter
\end{itemize}

