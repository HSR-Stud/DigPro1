\section{Digital image fundamentals \buch{p.35}}
\subsection{Image Interpolation \buch{p.65}}
\begin{itemize}
  \item Basic tool needed for zooming, shrinking, rotation, \ldots
  \item Nearest Neighbor Interpolation: 
	  \begin{itemize}
		  \item Each pixel in the new resolution gets the value of the nearest pixel in the old resolution.
		  \item Simple but bad
	\end{itemize}
  \item Bilinear interpolation:
  	\begin{itemize}
		\item \emph{Not} linear
  	  \item uses the four nearst neighbors of a point $(x,y)$
  	  \item $v(x,y) = ax + by + cxy + d$
  	\end{itemize}
  \item Bicubic interpolation
	\begin{itemize}
  	  \item Uses the 16 nearest neighbors of a point $(x,y)$
  	  \item Note that if the sums go from 0 to 1, this reduces to the bilinear interpolation
  	  \item $v(x,y) = \sum\limits_{i=0}^3\sum\limits_{j=0}^3 a_{ij}x^iy^j$
  	\end{itemize}
\end{itemize}


\subsection{Some Basic Relationships between pixels \buch{p.68}}
\subsubsection{Neighbors of a Pixel}
A pixel p at coordinates (x,y) has four horizontal and vertical neighors whose coordinates are given by
\[
	(x+1, y), (x-1, y), (x, y+1), (x, y-1)
\]
This set of pixels, called the 4-neighbors of p, is denoted by $N_4(p)$

The four diagonal neighbors of p have the coordinates
\[
	(x+1, y+1), (x+1, y-1), (x-1, y+1), (x-1, y-1)
\]
and are denoted by $N_D(p)$. $N_D(p)$ and $N_4(p)$ together are called the 8-neighbors of p, denoted by $N_8(p)$.


\subsubsection{Adjacency, connectivity, regions and boundaries}
\begin{description}
  \item[4-adjacency:] Two pixels p and q with values of V are 4-adjacency if q is in the set $N_4(p)$
  \item[8-adjacency:] Two pixels p and q with values of V are 8-adjacency if q is in the set $N_8(p)$
  \item[m-adjacency:] (mixed-adjancency) Two pixels p and q with values of V are m-adjacency if
  	\begin{enumerate}
  		\item q is in $N_4(p)$ or
  		\item q is in $N_D(p)$ and the set $N_4(p) \cap N_4(q)$ has no pixel whose values are from V 
	\end{enumerate}
\end{description}

\subsubsection{Distance measures / Neighbors \buch{p.71}}
For pixels $p,q$  with coordinates $(x,y), (s,t)$
\paragraph{Euclidean}
\begin{equation}
D_e(p,q) = [(x-s)^2 + (y-t)^2]^{\frac{1}{2}}
\end{equation}
\paragraph{City-block}
\begin{equation}
D_4(p,q) = |x-s| + |y-t|
\end{equation}
Pixels with a $D_4 = 1$ are the 4-neighbors of $(x,y)$
\paragraph{Chessboard}
\begin{equation}
D_8(p,q) = max(|x-s|, |y-t|)
\end{equation}
Pixels with a $D_8 = 1$ are the 8-neighbors of $(x,y)$

\subsubsection{Set and Logical Operations \buch{p.80}}
\begin{tabular}{|l|l|l|}
	\hline
	Subset			& $A \subseteq B$						& Every element of A is also in B
	\\ \hline
	Union			& $C = A \cup B$						& C contains all Elements in A, B or both
	\\ \hline
	Intersection	& $D = A \cap B$						& D contains all Elements wich are in A and B
	\\ \hline
	Complement		& $A^c = \{ \omega | \omega \in A\}$	& Set of elements that are not in A
	\\ \hline
	Difference		& $A-B = \{ \omega | \omega \in A, \omega \notin B\} = A \cap B^c$	&
	\\ \hline
\end{tabular}


\subsection{Math tools}
\subsubsection{Denoising}
Image with noise
\begin{equation}
g(x,y) = f(x,y) + \eta(x,y)
\end{equation}
Noise $\eta$ is uncorrelated and has zero average.  Averaging over $K$ different images reduces noise.
\begin{eqnarray}
\hat{g}(x,y)      =& \frac{1}{K} \sum_{i=1}^{K}g_i(x,y) \notag \\
E{\hat{g}(x,y)}   =& f(x,y) \\
\sigma_{g(x,y)}^2 =& \frac{1}{K} \sigma_{\eta(x,y)}^2 \notag
\end{eqnarray}

\subsubsection{Neighborhood operations}
Averaging
\begin{equation}
g(x,y) = \frac{1}{mn} \sum_{(r,c)\in S_{xy}}f(r,c)
\end{equation}

\subsubsection{Geometric Transformations}
\begin{eqnarray}
(x,y) =& T{(v,w)} \notag \\
 \left[ x~y~1 \right] =& [ v~w~q ] T = [ v~w~1 ] 
 \left[ \begin{array}{ccc}
t_{11} & t_{12} & 0 \\
t_{21} & t_{22} & 0 \\
t_{31} & t_{32} & 1 \end{array} \right]
\end{eqnarray}
\begin{itemize}
\item Spatial transformation of coordinates
\item Intensity interpolation for transformed pixels
\item See p.88 for examples of T!
\end{itemize}
